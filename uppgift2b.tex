\documentclass[12pt,a4paper]{article}

\usepackage[utf8]{inputenc}
\usepackage[T1]{fontenc}
\usepackage[swedish]{babel}
\usepackage{amsmath}
\usepackage[retainorgcmds]{IEEEtrantools}
\usepackage{epsfig}
\begin{document}
\section{Problem 2}
\subsection{Delproblem a}
Vi har ett symmetriaxel-fixt koordinatsystem $\xi\eta\zeta$ där $\mathbf{\hat{\zeta}}$ är huvudsymmetriaxeln, vari kroppen får tröghetsmatrisen: 
\begin{IEEEeqnarray*}{rCl} I & = & 
\begin{bmatrix}
I_o & 0 & 0 \\
0 & I_o & 0 \\
0 & 0 & I_{\zeta}
\end{bmatrix}
\end{IEEEeqnarray*}
Låt $\boldsymbol{\nu}$ vara kroppens rotation i $\xi\eta\zeta$, dess spinn. Kroppens absoluta rotaionshastighet blir då: $\boldsymbol{\omega}=\boldsymbol{\Omega}+\boldsymbol{\nu}$. Vi får genom insättning i Eulers ekvationer: $\mathbf{M}=\mathbf{\dot{L_{\xi\eta\zeta}}} + \mathbf{\Omega} \times \mathbf{L} = I\mathbf{\dot{\omega}} + \mathbf{\Omega} \times I\mathbf{\omega}$ Och för fallet då vi inte har några yttre vridmoment får vi $\mathbf{0}=I\mathbf{\dot{\omega}} + \mathbf{\Omega} \times I\mathbf{\omega}$, eller, komponentvis:
\begin{IEEEeqnarray}{rCl}
0 &=& I_o \dot{\omega}_{\xi} + I_{\zeta} \Omega_{\eta} \omega_{\zeta} - I_o \Omega_{\zeta} \omega_{\eta} \\
0 &=& I_o \dot{\omega}_{\eta} + I_o \Omega_{\zeta} \omega_{\xi} - I_{\zeta} \Omega_{\xi} \omega_{\zeta} \\
0 &=& I_{\zeta} \dot{\omega}_{\zeta} + I_o \Omega_{\xi} \omega_{\eta} - I_o \Omega_{\eta} \omega_{\xi}
\end{IEEEeqnarray}
 Då $\xi\eta\zeta$ är ett symmetriaxel-fixt så kommer kroppen inte ha någon rotation relativt systemet annat än i $\hat{\zeta}$-riktning, varpå vi får att $\nu_{\xi}=\nu_{\eta}=0$ och våra ekvationer blir:
\begin{IEEEeqnarray}{rCl}
0 &=& I_o \dot{\Omega}_{\xi} + I_{\zeta} \Omega_{\eta} (\Omega_{\zeta}+\nu_{\zeta}) - I_o \Omega_{\zeta} \Omega_{\eta} \\
0 &=& I_o \dot{\Omega}_{\eta} + I_o \Omega_{\zeta} \Omega_{\xi} - I_{\zeta} \Omega_{\xi} (\Omega_{\zeta} + \nu_{\zeta}) \\
0 &=& I_{\zeta}\dot{\Omega}_{\zeta} + I_{\zeta}\dot{\nu}_{\zeta}
\end{IEEEeqnarray}
Vilket ger: $\dot{\nu_{\zeta}}=-\dot{\Omega}_{\zeta}\, \Omega_{\xi}\dot{\Omega}_{\xi}=\Omega_{\eta}\dot{\Omega}_{\eta} $

När $\frac{d}{dt}(L_{\xi\eta\zeta})$ tolkades som derivatana rörelsemängdsmomentet såsom det verkar för någon som befinner sig i det roterande koordinatsystemet, dvs endast kroppens rotation $\boldsymbol{\nu}$ togs till hänsyn istället för $\boldsymbol{\omega}$ fick vi $\dot{\boldsymbol{\nu}} = \mathbf{0} , \nu_{\zeta} = \Omega_{\zeta}\frac{I_o-I_{\zeta}}{I_{\zeta}} - \frac{I_o \dot{\Omega}_{\xi}}{I_{\zeta} \Omega_{\eta}} , \Omega_{\zeta} = \frac{I_{\zeta} \nu_{\zeta}}{I_o-I_{\zeta}}$ , under förutsättningen att $\nu_{\xi}=\nu_{\eta}=0$. Båda av dessa antaganden bör tas med skeptism, liksom allt som står här.

\subsection{Delproblem b}
Då vi har kroppsfixt system utan yttre vridande moment bör precessionsvektorn gå igenom kroppens masscentrum.  Då vi enligt delproblem a har ett värde för precessionsvektorn i rotationsvektorns riktning, samt via trigonometri har dess riktning i förhållande till rotationvektorns riktning, så får vi precessionsvektorns belopp uttryckt i symmetriaxelfixt systemet $\hat{x}\hat{y}\hat{z}$ och därmed dess belopp:
\begin{equation*}
\Omega_z = \frac{\nu_z I_z}{I_o-I_z}
\Omega_x = \frac{b \nu_z I_z}{R(I_o-I_z)}
\abs{\boldsymbol{\Omega}}=\frac{\nu_z I_z R\cos(b/R)}{I_o-I_z}}}
\end{equation*}
Enligt förutsättningarna är det ett avstånd på 10 meter mellan rotation och precessionsvektor vid jordytan, $b=10m$. Radien vid polerna är $R_p= 6356,7523 km$, radien vid ekvatorn: $R_e=6378,1370 km$. För att få jordens tröghetsmoment slår vi inte upp tabellvärden utan integrerar över en ellipsoid. Fås att $I_o=\frac{m}{5}(R_p^2+R_e^2), I_z=\frac{2m}{5}R_e^2$ vilket ger relativa skillnaden: $\frac{I_o-I_z}{I_z} = -0.0033472$, samt att $\Omega \approx -300\nu$.Om $\omega=\Omega+\nu = \frac{2\pi}{24h}$ ger detta att $\nu \approx \frac{-\frac{2\pi}{300}}{24h}$, alltså spinner jorden ett varv på 300 dagar i motsatt riktning till dygn-snurr.
\\
\input{jord_t}

\subsection{Delproblem c}
För en generell precessionsrörelse $\boldsymbol{\Omega} = \Omega_x \hat{x} + \Omega_y \hat{y} + \Omega_z \hat{z}$ och en
\end{document}















%JORDENS tröghetsmoment hämtade från: http://scienceworld.wolfram.com/physics/MomentofInertiaEarth.html
%Jordens radie vid polerna från Wikipedia

%Formeln för tröghetsmomentet, homogen massfördelning: I_z=(2m/5)b^2, I_o=(m/5)(a^2+b^2)
%där b är radien vid ekvatorn, a är radien vid polerna
