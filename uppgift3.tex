\documentclass[12pt,a4paper]{article}

\usepackage[utf8]{inputenc}
\usepackage[T1]{fontenc}
\usepackage[swedish]{babel}
\usepackage{amsmath}
\usepackage[retainorgcmds]{IEEEtrantools}
\usepackage{color}
\usepackage{epsfig}
\begin{document}
Om vi har en ren vinkelhastighet i något led så fås ingen vinkelaccelleration i vår modell. Då $\gamma_y$ är mindre än noll så leder enligt ekvationerna vinkelhastigheter i x- och y-led till att $\dot{\omega_y}<0$ vilket över tid, då $\omega_y$ kommer in i de andra ekvationerna leder till att allt blir 0 då $\omega_y=0$ vilket ger oss konstant rotation. Om vi däremot inte har större vinkelhastigheter i x- och y-led så kommer $\dot{\omega_y}$ inte bli negativt nog och svängingsrörelsen blir oregelbunden. Problemet blir instabilt: små begynnelsevärden i x och y-led ger stora utslag. Detta överänsstämmer väll med det vi observerade.
Ändring i hastighetsriktning kommer från att att kroppen spinner runt och därför ändrar orientering; en observatör utanför koordinatsystemet skulle inte märka några plötsliga ändringar i hastighetens riktning.
\end{document}
